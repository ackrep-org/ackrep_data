\documentclass[10pt,a4paper]{article}
\usepackage[english]{babel}
\usepackage[utf8]{inputenc}
\usepackage{amsmath}
\usepackage{amsfonts}
\usepackage{amssymb}
\usepackage{graphicx}
\usepackage{float}
\usepackage{hyperref}

%link to documentation: 
%https://ackrep-doc.readthedocs.io/en/latest/devdoc/contributing_data.html

\begin{document}
	\part*{Model Documentation of the Heat Equation} % MUST - Add Model Name 
	
	%%%%%%%%%%%%%%%%%%%%%% NOMENCLATURE %%%%%%%%%%%%%%%%%%%%%%%%%%%
	
	\section{Nomenclature} % MUST
	\subsection{Nomenclature for Model Equations} % MUST
	
	%variables for model equations
	\begin{tabular}{ll}
		$t$ & time \\
		$z$ & space \\
		$\alpha$ & thermal diffusivity\\
		$u(z, t)$ & input trajectory \\
		$x(z, t)$ & wanted function describing spacial and temporal development of the temperature in matter
	\end{tabular}
	 
	
	%variables which are used additional to those in the model equations
	\begin{tabular}{ll}

	\end{tabular}
	
	%%%%%%%%%%%%%%%%%%%%%% MDOEL EQUATIONS %%%%%%%%%%%%%%%%%%%%%%%%%%%
	
	\section{Model Equations} % MUST
	
	\noindent System Equations:			
	\begin{subequations}
	\begin{align*}
		\dot{x}(z,t) = \alpha x''&(z,t) && z\in (0, l), t>0\\
		x(z,0) &= x_0(z) && z\in [0,l]\\
		x(0,t) &= 0 && t>0\\
		x(l,t) &= u(t) && t>0
	\end{align*}
	\end{subequations}

	%%%%%%%%%%%%%%%%%%%%%% PARAMETERS | OUTPUTS %%%%%%%%%%%%%%%%%%%%%%%%%%%
	\noindent
	Parameters: $\alpha, l,$ % variables with constant, predefined value

	
	%%%%%%%%%%%%%%%%%%%%%% ASSUMPTIONS %%%%%%%%%%%%%%%%%%%%%%%%%%%
	
	\subsection{Assumptions} % MAY 
		\begin{enumerate} %possible list type for the Assumptions
			\item $x_0(z)=0$
		\end{enumerate}
	
	%%%%%%%%%%%%%%%%%%%%%% EXEMPLARY PARAMETER VALUES %%%%%%%%%%%%%%%%%%%%%%%%%%%	
	
	\subsection{Exemplary parameter values}
	\begin{tabular}{cl}
\hline
  Symbol  & Value                                                                                                                                                                                \\
\hline
   $A$    & $\left[\begin{matrix}0.8189 & 0.0863 & 0.09 & 0.0813\\0.2524 & 1.0033 & 0.0313 & 0.2004\\-0.0545 & 0.0102 & 0.7901 & -0.258\\-0.1918 & -0.1034 & 0.1602 & 0.8604\end{matrix}\right]$ \\
   $B$    & $\left[\begin{matrix}0.0045 & 0.0044\\0.1001 & 0.01\\0.0003 & -0.0136\\-0.0051 & 0.0936\end{matrix}\right]$                                                                          \\
 $B_{1}$  & $\left[\begin{matrix}0.0045 & 0.0044\\0.1001 & 0.01\\0.0003 & -0.0136\\-0.0051 & 0.0936\end{matrix}\right]$                                                                          \\
 $C_{1}$  & $\left[\begin{matrix}1.0 & 0 & -1.0 & 0\\0 & 0 & 0 & 0\\0 & 0 & 0 & 0\end{matrix}\right]$                                                                                            \\
   $C$    & $\left[\begin{matrix}1.0 & 0 & 0 & 0\\0 & 0 & 1.0 & 0\end{matrix}\right]$                                                                                                            \\
 $D_{11}$ & $\left[\begin{matrix}0 & 0 & 0\\0 & 0 & 0\\0 & 0 & 0\end{matrix}\right]$                                                                                                             \\
 $D_{12}$ & $\left[\begin{matrix}0 & 0\\1.0 & 0\\0 & 1.0\end{matrix}\right]$                                                                                                                     \\
 $D_{21}$ & $\left[\begin{matrix}0 & 1.0 & 0\\0 & 0 & 1.0\end{matrix}\right]$                                                                                                                    \\
\hline
\end{tabular}

	%%%%%%%%%%%%%%%%%%%%%% DERIVATION & EXPLANATION %%%%%%%%%%%%%%%%%%%%%%%%%%%	
	
	\section{Derivation and Explanation} % SHOULD
	Approach: 
	\begin{itemize}
		\item inital functions $\varphi_1(z),...,\varphi_{n+1}(z)$
		\item test functions $\varphi_1(z),...,\varphi_n(z)$
		\item where the functions $\varphi_1(z),..,\varphi_n(z)$ met the homogeneous b.c.\\
			$\varphi_1(l),..,\varphi_n(l)=\varphi_1(0),..,\varphi_n(0)=0$
		\item only $\varphi_{n+1}$ can draw the actuation
	\end{itemize}
	Approximating the wanted function with
	\begin{equation*}
		x(z,t) = \sum_{i=1}^{n+1} x_i^*(t) \varphi_i(z)\Big|_{x^*_{n+1}=u} =  \underbrace{\sum_{i=1}^{n} x_i^*(t) \varphi_i(z)}_{\hat x(z,t)} + \varphi_{n+1}(z) u(t).
	\end{equation*}

	\noindent The weak formulation is given by

	\begin{align*}
		\langle\dot{x}(z,t),\varphi_j(z)\rangle &=
		a_2 \langle x''(z,t),\varphi_j(z)\rangle \\
		&\hphantom =+
		a_1 \langle x'(z,t), \varphi_j(z)\rangle  +
		a_0 \langle x(z,t), \varphi_j(z)\rangle && j=1,...,n.
	\end{align*}

	\noindent Shift of derivation to work with lagrange 1st order initial functions

	\begin{align*}
		\langle\dot{x}(z,t),\varphi_j(z)\rangle &=
		\overbrace{[a_2 [x'(z,t)\varphi_j(z)]_0^l}^{=0} - a_2 \langle x'(z,t),\varphi'_j(z)\rangle \\
		&\hphantom =+
		a_1 \langle x'(z,t), \varphi_j(z)\rangle  +
		a_0 \langle x(z,t), \varphi_j(z)\rangle && j=1,...,n \\
		\langle\dot{\hat{x}}(z,t),\varphi_j(z)\rangle + \langle\varphi_{N+1}(z),\varphi_j(z)\rangle \dot u(t) &= - a_2 \langle \hat x'(z,t),\varphi'_j(z)\rangle - a_2 \langle \varphi'_{N+1}(z),\varphi'_j(z)\rangle u(t) \\
		&\hphantom =+
		a_1 \langle \hat x'(z,t), \varphi_j(z)\rangle + a_1 \langle \varphi'_{N+1}(z), \varphi_j(z)\rangle u(t)  + \\
		&\hphantom =+
		a_0 \langle \hat x(z,t), \varphi_j(z)\rangle + a_0 \langle \varphi_{N+1}(z), \varphi_j(z)\rangle u(t) && j=1,...,n
	\end{align*}

	\noindent leads to state space model for the weights $\boldsymbol{x}^*=(x_1^*,...,x_n^*)^T$

	\begin{align*}
		\dot{\boldsymbol{x}}^*(t) = A \boldsymbol x^*(t) + \boldsymbol b_0 u(t) + \boldsymbol b_1 \dot{u}(t).
	\end{align*}

	\noindent The input derivative can be eliminated through the transformation
	\begin{equation*}
		\bar{\boldsymbol{x}}^* = \tilde A \boldsymbol x^* - \boldsymbol{b}_1 u
	\end{equation*}
	\noindent with e.g.: $\tilde A = I$, and leads to the state space model


	\begin{align*}
		\dot{\bar{\boldsymbol{x}}}^*(t) &= \tilde A A\tilde A^{-1} \bar{\boldsymbol{x}}^*(t) + \tilde A(A\boldsymbol b_1 + \boldsymbol b_0) u(t) \\
		&= \bar A \bar{\boldsymbol{x}}^*(t) +\bar{\boldsymbol{b}} u(t).
	\end{align*}




	
	%%%%%%%%%%%%%%%%%%%%%% REFERENCES %%%%%%%%%%%%%%%%%%%%%%%%%%%
	
	\begin{thebibliography}{10}		
		\bibitem{pyinduct}Stefan Ecklebe, Marcus Riesmeier: \\
		\href{https://pyinduct.readthedocs.io/en/master/examples/rad\_dirichlet\_fem.html}{https://pyinduct.readthedocs.io/en/master/examples/rad\_dirichlet\_fem.html}
	\end{thebibliography}

\end{document}

