\documentclass[10pt,a4paper]{article}
\usepackage[english]{babel}
\usepackage[utf8]{inputenc}
\usepackage{amsmath}
\usepackage{amsfonts}
\usepackage{amssymb}
\usepackage{graphicx}
\usepackage{float}
\usepackage{caption}
\usepackage{url}

%link to documentation: 
%https://ackrep-doc.readthedocs.io/en/latest/devdoc/contributing_data.html

\begin{document}
	\part*{Model Documentation of the \\ Underactuated Manipulator with Two Revolute Joints} % MUST - Add Model Name 
	
	%%%%%%%%%%%%%%%%%%%%%% NOMENCLATURE %%%%%%%%%%%%%%%%%%%%%%%%%%%
	
	\section{Nomenclature} % MUST
	\subsection{Nomenclature for Model Equations} % MUST
	
	%variables for model equations
	\begin{tabular}{ll}
		$s_i$ & distance from the joint to the center of gravity of link $i$, where $i = 1,2$ \\
		$m_i$ & mass of the link $i$, where $i = 1,2$ \\
		$J_i$ & moment of inertia of the link $i$, where $i = 1,2$ \\
		$l_1$ & length of link 1 \\
		$\tau$ & torque \\
		$q_1$ & angle between basis and link 1\\
		$p_1$ & angle between link 1 and link 2\\
		
	\end{tabular}
	 
	
	\subsection{Graphic of the Structure}	
	\begin{figure}[H]
		\centering
		\captionsetup{justification=centering, margin=1cm}
		\includegraphics[width=70mm]{two_manipulator.pdf}
		\caption{Structure of the manipulator. \\ \footnotesize{Source: Knoll, Carsten/ Betrachtetes System: unteraktuierten Zweigelenkmanipulator.}}
	\end{figure}
	
	%%%%%%%%%%%%%%%%%%%%%% MDOEL EQUATIONS %%%%%%%%%%%%%%%%%%%%%%%%%%%
	
	\section{Model Equations} % MUST
	
	State Vector and Input Vector:
	\begin{align*}
		\underline{x} &= (p_1 \ q_1 \ \dot{p}_1 \ \dot{q}_1)^T &= (x_1 \ x_2 \ x_3 \ x_4)^T \\
		u &= \tau 
	\end{align*}
	
	\noindent Kinetic Energy:			
	\begin{subequations}
	\begin{align*}
		T_{trans} &= \frac{1}{2} m_1s_1^2x_3^2 \sin^2 x_1+ \frac{1}{2} m_1s_1^2x_3^2 \cos^2 x_1 \\
		&+ \frac{m_2}{2}(-l_1x_3 \sin x_1 - s_2(x_3 + x_4) \sin(x_1 + x_2))^2 \\
		&+ \frac{m_2}{2}(l_1x_3 \cos x_1 + s_2(x_3 + x_4) \cos(x_1 + x_2))^2 \\
		T_{rot} &= \frac{1}{2} J_1x_3^2 + \frac{1}{2} J_2(x_3 + x_4)^2	
	\end{align*}
	\end{subequations}
		
	
	\noindent Potential Energy:			
	\begin{subequations}
	\begin{align*}
		V &= 0
	\end{align*}
	\end{subequations}

	%%%%%%%%%%%%%%%%%%%%%% PARAMETERS | OUTPUTS %%%%%%%%%%%%%%%%%%%%%%%%%%%
	\noindent
	Parameters: $s_1, \, s_2, \, m_1, \, m_2, \, J_1, \, J_2, \, l_1$ % variables with constant, predefined value
	\\
	Outputs: \underline{x} % MAY
	
	%%%%%%%%%%%%%%%%%%%%%% ASSUMPTIONS %%%%%%%%%%%%%%%%%%%%%%%%%%%
	
	\subsection{Assumptions} % MAY 
		\begin{enumerate} %possible list type for the Assumptions
			\item Friction is not taken into account. 
		\end{enumerate}
	
	%%%%%%%%%%%%%%%%%%%%%% EXEMPLARY PARAMETER VALUES %%%%%%%%%%%%%%%%%%%%%%%%%%%	
	
	\subsection{Exemplary parameter values}
	\begin{tabular}{cl}
\hline
  Symbol  & Value                                                                                          \\
\hline
   $A$    & $\left[\begin{matrix}0 & 1.0 & 0\\-79.285 & -0.113 & 0\\28.564 & 0.041 & 0\end{matrix}\right]$ \\
   $B$    & $\left[\begin{matrix}0 & 0\\0.041 & -0.0047\\-0.03 & -0.0016\end{matrix}\right]$               \\
 $B_{1}$  & $\left[\begin{matrix}0 & 0\\0.041 & -0.0047\\-0.03 & -0.0016\end{matrix}\right]$               \\
 $C_{1}$  & $\left[\begin{matrix}0 & 0 & 1.0\\1.0 & 0 & 0\\0 & 0 & 0\\0 & 0 & 0\end{matrix}\right]$        \\
   $C$    & $\left[\begin{matrix}0 & 0 & 1.0\\1.0 & 0 & 0\end{matrix}\right]$                              \\
 $D_{11}$ & $\left[\begin{matrix}0\\0\\0\\0\end{matrix}\right]$                                            \\
 $D_{12}$ & $\left[\begin{matrix}0 & 0\\0 & 0\\0.1 & 0\\0 & 0.1\end{matrix}\right]$                        \\
 $D_{21}$ & $\left[\begin{matrix}0\\0\end{matrix}\right]$                                                  \\
\hline
\end{tabular}

	%%%%%%%%%%%%%%%%%%%%%% DERIVATION & EXPLANATION %%%%%%%%%%%%%%%%%%%%%%%%%%%	
	
	\section{Derivation and Explanation} % SHOULD
	
	The Lagrangian mechanics was used for the solution.
	
	%%%%%%%%%%%%%%%%%%%%%% REFERENCES %%%%%%%%%%%%%%%%%%%%%%%%%%%
	
	\begin{thebibliography}{10}	
		\bibitem{ck1}Knoll, Carsten: 
		\textit{Betrachtetes System: unteraktuierten Zweigelenkmanipulator.}, Jupyter Notebook published 2016. \\
		\url{https://github.com/cknoll/beispiele/blob/master/zweigelenk_manipulator.ipynb}	
		\bibitem{But21}Wang, Yang: 
		\textit{Erstellung eines regelungstheoretischen Katalogs unteraktuierter mechanischer Systeme}, master thesis at the Institut of Control Theory TU Dresden, published 2016. \\
		(not publicly accessible)
	\end{thebibliography}

\end{document}

