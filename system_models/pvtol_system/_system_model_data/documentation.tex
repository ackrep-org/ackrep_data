\documentclass[10pt,a4paper]{article}
\usepackage[english]{babel}
\usepackage[utf8]{inputenc}
\usepackage{amsmath}
\usepackage{amsfonts}
\usepackage{amssymb}
\usepackage{graphicx}
\usepackage{float}

%< > ... dummy content zur Veranschaulichung

\begin{document}
	\part*{Model Documentation of the \\PVTOL with 2 Forces} % MUST - Add Model Name 
	
	%%%%%%%%%%%%%%%%%%%%%% NOMENCLATURE %%%%%%%%%%%%%%%%%%%%%%%%%%%
	
	\section{Nomenclature} % MUST
	\subsection{Nomenclature for Model Equations} % MUST
	
	%variables for model equations
	\begin{tabular}{ll}
		$x$ & horizontal displacement \\
		$y$ & vertical displacement \\
		$\theta$ & roll angle \\
		$F_1$, $F_2$ & Forces on the left and right site of the PVTOL \\
		$g$ & acceleration due to gravity \\
		$l$ & distance between mass center of PVTOL and target point of the forces \\
		$m$ & mass of the PVTOL \\
		$J$ & moment of inertia of the PVTOL		
	\end{tabular}

	
	%variables which are used additional to those in the model equations
	\begin{tabular}{ll}
	\end{tabular}
	
	%%%%%%%%%%%%%%%%%%%%%% MDOEL EQUATIONS %%%%%%%%%%%%%%%%%%%%%%%%%%%
	
	\section{Model Equations} % MUST
	
	State Vector and Input Vector:
	\begin{align*}
		\underline{x} &= (x_1 \ x_2 \ x_3 \ x_4 \ x_5 \ x_6)^T = (x \ \dot{x} \ y \ \dot{y} \ \theta \ \dot{\theta})^T \\
		\underline{u} &= (u_1 \ u_2)^T = (F_1 \ F_2)^T
	\end{align*}

	\noindent Model Equations:	
	\begin{subequations}
	\begin{align}
		\dot{x}_1 &= x_2 	\\ 
		\dot{x}_2 &= -\frac{\sin(x_5)}{m} (u_1 + u_2)  \\
		\dot{x}_3 &= x_4 \\
		\dot{x}_4 &= \frac{\cos(x_5)}{m} (u_1 + u_2) - g \\
		\dot{x}_5 &= x_6 \\
		\dot{x}_6 &= \frac{l}{J} (u_2 - u_1)
	\end{align}
	\end{subequations}

	%%%%%%%%%%%%%%%%%%%%%% INPUTS| PARAMETERS | OUTPUTS %%%%%%%%%%%%%%%%%%%%%%%%%%%
	\noindent
	Parameters: $m, ~J, ~l, ~g$ % variables with constant, predefined value
	\\
	Outputs:  $x, ~y, ~\theta$% MAY
	
	%%%%%%%%%%%%%%%%%%%%%% ASSUMPTIONS %%%%%%%%%%%%%%%%%%%%%%%%%%%
	
	\subsection{Assumptions} % MAY 
		\begin{enumerate} %possible list type for the Assumptions - mögliche Formatierung für die Annahmen
			\item forces target the body of the PVTOL in a 90° angle
		\end{enumerate}
	
	%%%%%%%%%%%%%%%%%%%%%% EXEMPLARY PARAMETER VALUES %%%%%%%%%%%%%%%%%%%%%%%%%%%	
	
	\subsection{Exemplary parameter values}
	\begin{tabular}{cl}
\hline
  Symbol  & Value                                                                                          \\
\hline
   $A$    & $\left[\begin{matrix}0 & 1.0 & 0\\-79.285 & -0.113 & 0\\28.564 & 0.041 & 0\end{matrix}\right]$ \\
   $B$    & $\left[\begin{matrix}0 & 0\\0.041 & -0.0047\\-0.03 & -0.0016\end{matrix}\right]$               \\
 $B_{1}$  & $\left[\begin{matrix}0 & 0\\0.041 & -0.0047\\-0.03 & -0.0016\end{matrix}\right]$               \\
 $C_{1}$  & $\left[\begin{matrix}0 & 0 & 1.0\\1.0 & 0 & 0\\0 & 0 & 0\\0 & 0 & 0\end{matrix}\right]$        \\
   $C$    & $\left[\begin{matrix}0 & 0 & 1.0\\1.0 & 0 & 0\end{matrix}\right]$                              \\
 $D_{11}$ & $\left[\begin{matrix}0\\0\\0\\0\end{matrix}\right]$                                            \\
 $D_{12}$ & $\left[\begin{matrix}0 & 0\\0 & 0\\0.1 & 0\\0 & 0.1\end{matrix}\right]$                        \\
 $D_{21}$ & $\left[\begin{matrix}0\\0\end{matrix}\right]$                                                  \\
\hline
\end{tabular}

	%%%%%%%%%%%%%%%%%%%%%% DERIVATION & EXPLANATION %%%%%%%%%%%%%%%%%%%%%%%%%%%	
	
	\section{Derivation and Explanation} % SHOULD
	
	\textit{Not available}
	%%%%%%%%%%%%%%%%%%%%%% REFERENCES %%%%%%%%%%%%%%%%%%%%%%%%%%%
	
	\begin{thebibliography}{10}		
		\bibitem{KNOLL16}Knoll, C: 
		\textit{Regelungstheoretische Analyse- und Entwurfsansätze für unteraktuierte mechanische Systeme}, p. 169, TU Dresden, 2016
	\end{thebibliography}

\end{document}