\documentclass[10pt,a4paper]{article}
\usepackage[english]{babel}
\usepackage[utf8]{inputenc}
\usepackage{amsmath}
\usepackage{amsfonts}
\usepackage{amssymb}
\usepackage{graphicx}
\usepackage{float}
\usepackage{caption}

%link to documentation: 
%https://ackrep-doc.readthedocs.io/en/latest/devdoc/contributing_data.html

\begin{document}
	\part*{Model Documentation of the \\ Translational Oscillator with Rotational Actuator  (TORA)} % MUST - Add Model Name 
	
	%%%%%%%%%%%%%%%%%%%%%% NOMENCLATURE %%%%%%%%%%%%%%%%%%%%%%%%%%%
	
	\section{Nomenclature} % MUST
	\subsection{Nomenclature for Model Equations} % MUST
	
	%variables for model equations
	\begin{tabular}{ll}
		$m_1$ & mass of the cart \\
		$m_2$ & mass of the pendulum \\
		$l_1$ & length of the pendulum \\
		$J_1$ & moment of inertia of the pendulum \\	
		$\alpha$ & spring constant \\
		$\tau$ & torque \\
		$q_1$ & position of the cart \\
		$q_2$ & angel of the pendulum \\
				
	\end{tabular}
	 
	\subsection{Graphic of the Structure}	
	\begin{figure}[H]
		\centering
		\captionsetup{justification=centering, margin=1cm}
		\includegraphics[width=70mm]{tora.pdf}
		\caption{Structure of the TORA. \\ \footnotesize{Source: Wang, Yang/Erstellung eines regelungstheoretischen Katalogs unteraktuierter mechanischer Systeme}}
	\end{figure}
	
	%%%%%%%%%%%%%%%%%%%%%% MDOEL EQUATIONS %%%%%%%%%%%%%%%%%%%%%%%%%%%
	
	\section{Model Equations} % MUST
	
	State Vector and Input Vector:
	\begin{align*}
		\underline{x} &= (q_1 \ q_2 \ \dot{q}_1 \ \dot{q}_2)^T &= (x_1 \ x_2 \ x_3 \ x_4)^T \\
		u &= \tau 
	\end{align*}
	
	\noindent System Equations:			
	\begin{subequations}
	\begin{align}
		\dot{x}_1 &= x_3 \\
		\dot{x}_2 &= x_4 \\
		\dot{x}_3 &= \frac{(m_2l_1^2 + J_1)(-\alpha x_1 + m_2l_1x_4^2 \sin x_2) - m_2l_1 \cos x_2 u}{(m_1 + m_2)(m_2l_1^2 + J_1) - m_2^2l_1^2 \cos^2 x_2} \\
		\dot{x}_4 &= \frac{-m_2l_1 \cos x_2(-\alpha x_1 + m_2l_1x_4^2 \sin x_2) + (m_1 + m_2)u}{(m_1 + m_2)(m_2l_1^2 + J_1) - m_2^2l_1^2 \cos^2 x_2}		
	\end{align}
	\end{subequations}

	%%%%%%%%%%%%%%%%%%%%%% PARAMETERS | OUTPUTS %%%%%%%%%%%%%%%%%%%%%%%%%%%
	\noindent
	Parameters: $m_1, \, m_2, \, l_1, \, J_1, \, \alpha$ % variables with constant, predefined value
	\\
	Outputs: \underline{x}
	
	%%%%%%%%%%%%%%%%%%%%%% EXEMPLARY PARAMETER VALUES %%%%%%%%%%%%%%%%%%%%%%%%%%%	
	
	\subsection{Exemplary parameter values}
	\begin{tabular}{cl}
\hline
  Symbol  & Value                                                                                          \\
\hline
   $A$    & $\left[\begin{matrix}0 & 1.0 & 0\\-79.285 & -0.113 & 0\\28.564 & 0.041 & 0\end{matrix}\right]$ \\
   $B$    & $\left[\begin{matrix}0 & 0\\0.041 & -0.0047\\-0.03 & -0.0016\end{matrix}\right]$               \\
 $B_{1}$  & $\left[\begin{matrix}0 & 0\\0.041 & -0.0047\\-0.03 & -0.0016\end{matrix}\right]$               \\
 $C_{1}$  & $\left[\begin{matrix}0 & 0 & 1.0\\1.0 & 0 & 0\\0 & 0 & 0\\0 & 0 & 0\end{matrix}\right]$        \\
   $C$    & $\left[\begin{matrix}0 & 0 & 1.0\\1.0 & 0 & 0\end{matrix}\right]$                              \\
 $D_{11}$ & $\left[\begin{matrix}0\\0\\0\\0\end{matrix}\right]$                                            \\
 $D_{12}$ & $\left[\begin{matrix}0 & 0\\0 & 0\\0.1 & 0\\0 & 0.1\end{matrix}\right]$                        \\
 $D_{21}$ & $\left[\begin{matrix}0\\0\end{matrix}\right]$                                                  \\
\hline
\end{tabular}

	%%%%%%%%%%%%%%%%%%%%%% DERIVATION & EXPLANATION %%%%%%%%%%%%%%%%%%%%%%%%%%%	
	
	\section{Derivation and Explanation} % SHOULD
	
	The Lagrangian mechanics was used for the solution.
	
	%%%%%%%%%%%%%%%%%%%%%% REFERENCES %%%%%%%%%%%%%%%%%%%%%%%%%%%
	
	\begin{thebibliography}{10}		
		\bibitem{But21}C.-J. Wan, D. Bernstein und V. Coppola: 
		\textit{Global stabilization of the oscillating
eccentric rotor.} In: \textit{Decision and Control, 1994., Proceedings of the 33rd IEEE
Conference on}, Bd. 4, S. 4024–4029 vol.4, Dec 1994.
		\bibitem{But21}Wang, Yang:  
		\textit{Erstellung eines regelungstheoretischen Katalogs unteraktuierter mechanischer Systeme}, master thesis at the Institut of Control Theory TU Dresden, published 2016. \\
		(not publicly accessible)
	\end{thebibliography}

\end{document}

