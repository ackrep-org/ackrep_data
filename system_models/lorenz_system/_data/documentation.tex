\documentclass[10pt,a4paper]{article}
\usepackage[english]{babel}
\usepackage[utf8]{inputenc}
\usepackage{amsmath}
\usepackage{amsfonts}
\usepackage{amssymb}
\usepackage{textcomp}
\usepackage{graphicx}
\usepackage{float}

%< > ... dummy content zur Veranschaulichung

\begin{document}
	\part*{Model Documentation of the \\ Lorentz Attractor} % MUST - Add Model Name 
	
	%%%%%%%%%%%%%%%%%%%%%% NOMENCLATURE %%%%%%%%%%%%%%%%%%%%%%%%%%%
	
	\section{Nomenclature} % MUST
	\subsection{Nomenclature for Model Equations} % MUST
	
	%variables for model equations
	\begin{tabular}{ll}
		$x$ & is proportional to convection motions \\
		$y$ & is proportional to temperature difference between ascending and descending currents\\
		$z$ & is proportional to distortion of vertical temperature profile from linearity\\
		$\sigma$ & Prandtl Number \\
		$r$ & quotient of Raileigh Number and a critical value (see \cite{LOR63})\\
		$b$ & parameter
	\end{tabular}
	
	%%%%%%%%%%%%%%%%%%%%%% MDOEL EQUATIONS %%%%%%%%%%%%%%%%%%%%%%%%%%%
	
	\section{Model Equations} % MUST
	State Vector and Input Vector:
	\begin{align*}
		\underline{x} &= (x_1 \ x_2 \ x_3)^T = (x \ y \ z )^T \\
		\underline{u} &= \emptyset
	\end{align*}

	\noindent Model Equations:
	\begin{subequations}
	\begin{align}
		\dot{x} &= -\sigma x + \sigma y 	\\      % possible ways to write d/dt
		\dot{y} &= -xz + rx - y 	\\
		\dot{z} &= xy - bz	
	\end{align}
	\end{subequations}

	%%%%%%%%%%%%%%%%%%%%%% PARAMETERS | OUTPUTS %%%%%%%%%%%%%%%%%%%%%%%%%%%
	
	\noindent
	Parameters: $\sigma,~r,~b$ % variables with constant, predefined value
	\\
	Outputs: \textit{\textlangle not defined\textrangle} 
	
	%%%%%%%%%%%%%%%%%%%%%% ASSUMPTIONS %%%%%%%%%%%%%%%%%%%%%%%%%%%	
	
	%%%%%%%%%%%%%%%%%%%%%% EXEMPLARY PARAMETER VALUES %%%%%%%%%%%%%%%%%%%%%%%%%%%	
	
	\subsection{Exemplary parameter values}
	\begin{tabular}{cl}
\hline
  Symbol  & Value                                                                                                                                                                                \\
\hline
   $A$    & $\left[\begin{matrix}0.8189 & 0.0863 & 0.09 & 0.0813\\0.2524 & 1.0033 & 0.0313 & 0.2004\\-0.0545 & 0.0102 & 0.7901 & -0.258\\-0.1918 & -0.1034 & 0.1602 & 0.8604\end{matrix}\right]$ \\
   $B$    & $\left[\begin{matrix}0.0045 & 0.0044\\0.1001 & 0.01\\0.0003 & -0.0136\\-0.0051 & 0.0936\end{matrix}\right]$                                                                          \\
 $B_{1}$  & $\left[\begin{matrix}0.0045 & 0.0044\\0.1001 & 0.01\\0.0003 & -0.0136\\-0.0051 & 0.0936\end{matrix}\right]$                                                                          \\
 $C_{1}$  & $\left[\begin{matrix}1.0 & 0 & -1.0 & 0\\0 & 0 & 0 & 0\\0 & 0 & 0 & 0\end{matrix}\right]$                                                                                            \\
   $C$    & $\left[\begin{matrix}1.0 & 0 & 0 & 0\\0 & 0 & 1.0 & 0\end{matrix}\right]$                                                                                                            \\
 $D_{11}$ & $\left[\begin{matrix}0 & 0 & 0\\0 & 0 & 0\\0 & 0 & 0\end{matrix}\right]$                                                                                                             \\
 $D_{12}$ & $\left[\begin{matrix}0 & 0\\1.0 & 0\\0 & 1.0\end{matrix}\right]$                                                                                                                     \\
 $D_{21}$ & $\left[\begin{matrix}0 & 1.0 & 0\\0 & 0 & 1.0\end{matrix}\right]$                                                                                                                    \\
\hline
\end{tabular}

	%%%%%%%%%%%%%%%%%%%%%% DERIVATION & EXPLANATION %%%%%%%%%%%%%%%%%%%%%%%%%%%	
	
	\section{Derivation and Explanation} % SHOULD
	
	The Lorenz Attractor is derived from the dynamic of convection currents. It is based on a model of convection currents formulated by Lord Rayleigh. 
	
	%%%%%%%%%%%%%%%%%%%%%% REFERENCES %%%%%%%%%%%%%%%%%%%%%%%%%%%
	
	\begin{thebibliography}{10}		
		\bibitem{LOR63}Lorenz, E. N.: 
		\textit{Deterministic Nonperiodic Flow}, p. 135, Journal of Atmospheric Sciences 1963.
	\end{thebibliography}

\end{document}
