\documentclass[10pt,a4paper]{article}
\usepackage[english]{babel}
\usepackage[utf8]{inputenc}
\usepackage{amsmath}
\usepackage{amsfonts}
\usepackage{amssymb}
\usepackage{graphicx}
\usepackage{float}
\usepackage{caption}

%link to documentation: 
%https://ackrep-doc.readthedocs.io/en/latest/devdoc/contributing_data.html

\begin{document}
	\part*{Model Documentation of the \\ Furuta Pendulum} % MUST - Add Model Name 
	
	%%%%%%%%%%%%%%%%%%%%%% NOMENCLATURE %%%%%%%%%%%%%%%%%%%%%%%%%%%
	
	\section{Nomenclature} % MUST
	\subsection{Nomenclature for Model Equations} % MUST
	
	%variables for model equations
	\begin{tabular}{ll}
		$m_2$ & mass of the pendulum \\
		$l_1$ & length of the arm \\		
		$l_2$ & length of the pendulum \\
		$J_1$ & moment of inertia of the arm \\
		$J_2$ & moment of inertia of the pendulum about the axis of rotation \\
		& through the center of mass \\
		$g$ & acceleration due to gravity \\
		$q_1$ & angel of the arm \\
		$q_2$ & angel of the pendulum \\
		$\tau$ & torque on the arm\\	
	\end{tabular}
	 
	
	\subsection{Graphic of the Structure}	
	\begin{figure}[H]
		\centering
		\captionsetup{justification=centering, margin=1cm}
		\includegraphics[width=70mm]{furuta.pdf}
		\caption{Structure of the Furuta Pendulum. \\ \footnotesize{Source: Wang, Yang/Erstellung eines regelungstheoretischen Katalogs unteraktuierter mechanischer Systeme}}
	\end{figure}
	
	%%%%%%%%%%%%%%%%%%%%%% MDOEL EQUATIONS %%%%%%%%%%%%%%%%%%%%%%%%%%%
	
	\section{Model Equations} % MUST
	
	State Vector and Input Vector:
	\begin{align*}
		\underline{x} &= (q_1 \ q_2 \ \dot{q}_1 \ \dot{q}_2)^T &= (x_1 \ x_2 \ x_3 \ x_4)^T \\
		u &= \tau\\
	\end{align*}
	
	\noindent Kinetic Energy:			
	\begin{subequations}
	\begin{align}
		T = \frac{1}{2}J_2x_3^2 + \frac{1}{2}m_2[(l_1^2 + l_2^2 \sin^2 x_2)x_3^2 + l_2^2x_4^2 + 2l_1l_2 \cos x_2x_3x_4] \\
	\end{align}
	\end{subequations}
	
	\noindent Potential Energy:			
	\begin{subequations}
	\begin{align}
		V = m_2gl_2(\cos x_2 - 1) \\
	\end{align}
	\end{subequations}

	%%%%%%%%%%%%%%%%%%%%%% PARAMETERS | OUTPUTS %%%%%%%%%%%%%%%%%%%%%%%%%%%
	\noindent
	Parameters: $m_2, \, l_1, \, l_2, \, J_1, \, J_2, \, g$ % variables with constant, predefined value
	\\
	Outputs: \underline{x}

	
	%%%%%%%%%%%%%%%%%%%%%% EXEMPLARY PARAMETER VALUES %%%%%%%%%%%%%%%%%%%%%%%%%%%	
	
	\subsection{Exemplary parameter values}
	\begin{tabular}{cl}
\hline
  Symbol  & Value                                                                                          \\
\hline
   $A$    & $\left[\begin{matrix}0 & 1.0 & 0\\-79.285 & -0.113 & 0\\28.564 & 0.041 & 0\end{matrix}\right]$ \\
   $B$    & $\left[\begin{matrix}0 & 0\\0.041 & -0.0047\\-0.03 & -0.0016\end{matrix}\right]$               \\
 $B_{1}$  & $\left[\begin{matrix}0 & 0\\0.041 & -0.0047\\-0.03 & -0.0016\end{matrix}\right]$               \\
 $C_{1}$  & $\left[\begin{matrix}0 & 0 & 1.0\\1.0 & 0 & 0\\0 & 0 & 0\\0 & 0 & 0\end{matrix}\right]$        \\
   $C$    & $\left[\begin{matrix}0 & 0 & 1.0\\1.0 & 0 & 0\end{matrix}\right]$                              \\
 $D_{11}$ & $\left[\begin{matrix}0\\0\\0\\0\end{matrix}\right]$                                            \\
 $D_{12}$ & $\left[\begin{matrix}0 & 0\\0 & 0\\0.1 & 0\\0 & 0.1\end{matrix}\right]$                        \\
 $D_{21}$ & $\left[\begin{matrix}0\\0\end{matrix}\right]$                                                  \\
\hline
\end{tabular}

	%%%%%%%%%%%%%%%%%%%%%% DERIVATION & EXPLANATION %%%%%%%%%%%%%%%%%%%%%%%%%%%	
	
	\section{Derivation and Explanation} % SHOULD
	
	The Lagrangian mechanics was used for the solution.
	
	%%%%%%%%%%%%%%%%%%%%%% REFERENCES %%%%%%%%%%%%%%%%%%%%%%%%%%%
	
	\begin{thebibliography}{10}	
		\bibitem{bib1}K. Furuta: 
		\textit{Swing-up control of inverted pendulum using pseudo-state feedback.}, Journal of Systems and Control Engineering, S. 263–269, published 1992.
		\bibitem{bib2}Wang, Yang:  
		\textit{Erstellung eines regelungstheoretischen Katalogs unteraktuierter mechanischer Systeme}, master thesis at the Institut of Control Theory TU Dresden, published 2016. \\
		(not publicly accessible)
	\end{thebibliography}

\end{document}

