\documentclass[10pt,a4paper]{article}
\usepackage[english]{babel}
\usepackage[utf8]{inputenc}
\usepackage{amsmath}
\usepackage{amsfonts}
\usepackage{amssymb}
\usepackage{graphicx}
\usepackage{float}

%link to documentation: 
%https://ackrep-doc.readthedocs.io/en/latest/devdoc/contributing_data.html

\begin{document}
	\part*{Model Documentation of the \\ Buck Converter} % MUST - Add Model Name 
	
	%%%%%%%%%%%%%%%%%%%%%% NOMENCLATURE %%%%%%%%%%%%%%%%%%%%%%%%%%%
	
	\section{Nomenclature} % MUST
	\subsection{Nomenclature for Model Equations} % MUST
	
	%variables for model equations
	\begin{tabular}{ll}
		$L$ & inductivity of the inductor \\
		$C$ & capacity of the capacitor \\
		$R$ & resistance of the load \\
		$U_E$ & input voltage \\
		$i_L$ & current through the inductor \\
		$u_C$ & voltage over the capatitor \\
		$d$ & duty ratio of the switch \\

		
				
	\end{tabular}
	 
	
	%%%%%%%%%%%%%%%%%%%%%% MDOEL EQUATIONS %%%%%%%%%%%%%%%%%%%%%%%%%%%
	
	\section{Model Equations} % MUST
	
	State Vector and Input Vector:
	\begin{align*}
		\underline{x} &= (x_1 \ x_2)^T = (i_L \ u_C)^T \\
		\underline{u} &= d
	\end{align*}
	
	\noindent System Equations:			
	\begin{subequations}
	\begin{align}
		\dot{x}_1 &= - \frac{1}{L}x_2+\frac{U_E}{L}u \\
		\dot{x}_2 &= \frac{1}{C}x_1 - \frac{1}{RC}x_2
	\end{align}
	\end{subequations}

	%%%%%%%%%%%%%%%%%%%%%% PARAMETERS | OUTPUTS %%%%%%%%%%%%%%%%%%%%%%%%%%%
	\noindent
	Parameters: $L, C, R, U_E$ % variables with constant, predefined value
	\\
	Outputs: $x_2 = u_C$
	
	%%%%%%%%%%%%%%%%%%%%%% ASSUMPTIONS %%%%%%%%%%%%%%%%%%%%%%%%%%%
	
	% \subsection{Assumptions} % MAY 
	% 	\begin{enumerate} %possible list type for the Assumptions
	% 		\item The switching frequency is high enough, to prevent the inductor from fully discharging beween charging stages.
	% 	\end{enumerate}
	
	%%%%%%%%%%%%%%%%%%%%%% EXEMPLARY PARAMETER VALUES %%%%%%%%%%%%%%%%%%%%%%%%%%%	
	
	\subsection{Exemplary parameter values}
	\begin{tabular}{cl}
\hline
  Symbol  & Value                                                                                                                                                                                \\
\hline
   $A$    & $\left[\begin{matrix}0.8189 & 0.0863 & 0.09 & 0.0813\\0.2524 & 1.0033 & 0.0313 & 0.2004\\-0.0545 & 0.0102 & 0.7901 & -0.258\\-0.1918 & -0.1034 & 0.1602 & 0.8604\end{matrix}\right]$ \\
   $B$    & $\left[\begin{matrix}0.0045 & 0.0044\\0.1001 & 0.01\\0.0003 & -0.0136\\-0.0051 & 0.0936\end{matrix}\right]$                                                                          \\
 $B_{1}$  & $\left[\begin{matrix}0.0045 & 0.0044\\0.1001 & 0.01\\0.0003 & -0.0136\\-0.0051 & 0.0936\end{matrix}\right]$                                                                          \\
 $C_{1}$  & $\left[\begin{matrix}1.0 & 0 & -1.0 & 0\\0 & 0 & 0 & 0\\0 & 0 & 0 & 0\end{matrix}\right]$                                                                                            \\
   $C$    & $\left[\begin{matrix}1.0 & 0 & 0 & 0\\0 & 0 & 1.0 & 0\end{matrix}\right]$                                                                                                            \\
 $D_{11}$ & $\left[\begin{matrix}0 & 0 & 0\\0 & 0 & 0\\0 & 0 & 0\end{matrix}\right]$                                                                                                             \\
 $D_{12}$ & $\left[\begin{matrix}0 & 0\\1.0 & 0\\0 & 1.0\end{matrix}\right]$                                                                                                                     \\
 $D_{21}$ & $\left[\begin{matrix}0 & 1.0 & 0\\0 & 0 & 1.0\end{matrix}\right]$                                                                                                                    \\
\hline
\end{tabular}

	%%%%%%%%%%%%%%%%%%%%%% DERIVATION & EXPLANATION %%%%%%%%%%%%%%%%%%%%%%%%%%%	
	
	\section{Derivation and Explanation} % SHOULD
	Using PWM (puls width modulation), instead of only discrete values $d\in\{0,1\}$ representing an \textit{open} or \textit{closed} switch, 
	any value of the interval $[0,1]$ can be modeled. This is done by using the averaged values for states and inputs:
	\begin{align*}
		\bar{d} &= \frac{1}{T}\int_t^{t+T} d(\tau)d\tau \\
		\bar{x}_i &= \frac{1}{T}\int_t^{t+T} x_i(\tau)d\tau \quad i=1,2
	\end{align*}
	with the switching period $T$. For $T\rightarrow 0$, which is achieved by a high enough switching frequency, an averaged model can be obtained.
	
	%%%%%%%%%%%%%%%%%%%%%% REFERENCES %%%%%%%%%%%%%%%%%%%%%%%%%%%
	
	\begin{thebibliography}{10}		
		\bibitem{TanHoo15}R. H. G. Tan and L. Y. H. Hoo, “DC-DC converter modeling and simulation using state space approach,” in 2015 IEEE Conference on Energy Conversion (CENCON), Oct. 2015, pp. 42–47. doi: 10.1109/CENCON.2015.7409511.
		\bibitem{Röb17}K. Röbenack, Nichtlineare Regelungssysteme: Theorie und Anwendung der exakten Linearisierung. Berlin, Heidelberg: Springer Berlin Heidelberg, 2017. doi: 10.1007/978-3-662-44091-9.
	\end{thebibliography}

\end{document}

